\documentclass[11pt]{article}

\usepackage[a4paper]{geometry}
\usepackage{amssymb} 
\usepackage{amsmath} 
\usepackage{euscript} 
\usepackage{latexsym}
\usepackage{mathrsfs}
\usepackage{stmaryrd}
\usepackage{graphicx}
\usepackage{dsfont}
\usepackage{proof}
\usepackage{color}
\usepackage{etex,xy}
\usepackage{version}
\xyoption{all} 

\includeversion{question}
\includeversion{solution}

\newcommand{\ipc}{{\sf IPC}}
\newcommand{\cpc}{{\sf CPC}}
\newcommand{\iqc}{{\sf IQC}}
\newcommand{\cqc}{{\sf CQC}}
\newcommand{\LJ}{{\sf LJ}}
\newcommand{\Gth}{{\sf G3}}
\newcommand{\GKp}{{\sf GKp}}
\newcommand{\LK}{{\sf LK}}
\newcommand{\K}{{\sf K}}
\newcommand{\KT}{{\sf KT}}
\newcommand{\Kf}{{\sf K4}}
\newcommand{\GL}{{\sf GL}}
\newcommand{\Sf}{{\sf S4}}

\newcommand{\T}{{T}}
\newcommand{\lang}{{\cal L}}
\newcommand{\lgc}{{\sf L}}

\newcommand{\langprop}{{\cal L}_{\rm prp}}
\newcommand{\langpred}{{\cal L}_{\rm prd}}
\newcommand{\formprop}{{\cal F}_{\rm prp}}
\newcommand{\formpred}{{\cal F}_{\rm prd}}

\newcommand{\af}{\vdash}
\newcommand{\waar}{\vDash}
\newcommand{\defn}{\equiv}
\newcommand{\slot}{\trs{$\Box$}}

\newcommand{\cala}{{\cal A}}
\newcommand{\calc}{{\cal C}}
\newcommand{\cald}{{\cal D}}
\newcommand{\mg}{{\cal G}}
\newcommand{\mh}{{\cal H}}
\newcommand{\cali}{{\cal I}}
\newcommand{\calp}{{\cal P}}
\newcommand{\calr}{{\cal R}}
\newcommand{\cals}{{\cal S}}
\newcommand{\calv}{{\cal V}}
\newcommand{\Ga}{\Gamma}
\newcommand{\The}{\Theta}  
\newcommand{\De}{\Delta} 
\newcommand{\Sig}{\Sigma}  
\newcommand{\ph}{\varphi}
\newcommand{\sig}{\sigma}

\newcommand{\imp}{\rightarrow} 
\newcommand{\Imp}{\, \Rightarrow \, }
\newcommand{\ifff}{\leftrightarrow} 
\newcommand{\Ifff}{\, \Leftrightarrow \, }
\newcommand{\Rgt}{\Rightarrow}
\newcommand{\Lft}{\Leftarrow}
\newcommand{\en}{\wedge} 
\newcommand{\of}{\vee} 
\newcommand{\E}{\exists}
\newcommand{\A}{\forall} 
\newcommand{\bof}{\bigvee}
\newcommand{\ben}{\bigwedge}
\newcommand{\bx}{\Box}
\newcommand{\dbx}{\boxdot}
\newcommand{\seq}{\Rightarrow}
\newcommand{\klg}{\leqslant}

\newcommand{\nd}[2]{#1 \vdash #2}
\newcommand{\sq}[2]{#1 \Rightarrow #2}
\newcommand{\Ra}{\rightarrow}
\newcommand{\Zip}[1]{\langle #1 \rangle}
\newcommand{\bang}{\mathop{!}}
\newcommand{\lolli}{\multimap}
\newcommand{\bs}{\backslash}
\newcommand{\with}{\mathbin{\&}}
\newcommand{\amp}{\mathbin{\&}}
\newcommand{\W}[1]{\textrm{#1}}
\newcommand{\fdia}{\diamondsuit}
\newcommand{\gbox}{\Box}
\newcommand\rlambda{\overset{{}_{\shortrightarrow}}{\lambda}}
\newcommand\llambda{\overset{{}_{\shortleftarrow}}{\lambda}}

\parindent0pt

\begin{document}
\setlength{\unitlength}{1cm}

\begin{center}
{\bf Methods in AI Research} \\
\vskip5pt 
{\bf Exercises II} \\
\vskip5pt 
\end{center}

\section{}
\begin{small}
$(\dag)\qquad$ 
\[
\infer[\lolli E]{\nd{\Zip{A},\Zip{B},\Zip{A\lolli (B\lolli C)}}{C}}
{\infer[\lolli I]{\nd{\Zip{A\lolli (B\lolli C)}}{(A\otimes B)\lolli C}}
{\infer[\otimes E]{\nd{\Zip{A \lolli (B \lolli C)}}{(A \otimes B)\lolli C}}{{\infer[]{\nd{(A \otimes B)}{(A \otimes B)}}{Ax}} & 
{\infer[\lolli E]{\nd{\Zip{A \lolli (B \lolli C)},{A},{B}}{C}}{
{\infer[\lolli E]{\nd{\Zip{A \lolli (B \lolli C)},A}{(B \lolli C)}}
{\infer[]{\nd{\Zip{A\lolli(B \lolli C)}}{\Zip{A \lolli (B \lolli C)}}}{Ax}}}
&{\infer[]{\nd{B}{B}}{Ax}}}}}} &
\infer[]{\nd{\Zip{A},\Zip{B}}{A\otimes B}}
{{\infer[]{\nd{\Zip{A}}{A}}{Ax}} &
{\infer[]{\nd{\Zip{B}}{B}}{Ax}}}}
\]

\[
\infer[\lolli E]{\nd{\Zip{A},\Zip{B},\Zip{A \lolli (B \lolli C)}}{C}}
{{\infer[E \lolli]{\nd{\Zip{A \lolli (B \lolli C)},\Zip{A}}{(B \lolli C)}}{{w}&{r}}}&
{\infer[]{\nd{B}{B}}{Ax}}}
\]
\end{small}
\begin{itemize}
  
  \item[-] the proof is not in normal form: perform the applicable
  proof reduction(s) to bring it in normal form;
  \item[-] compute the term for (\dag), and the term that results from term reduction(s).
\end{itemize}
You may leave applications of Exchange implicit, i.e.~treat the antecedent as
a multiset of assumptions. Remember: the normal form is reached when \emph{no further
reductions} are applicable.


\section{}
Below you find two ILL judgements and two programs (terms). Which 
term goes with which judgement?

\begin{enumerate}
\item $\lambda\Zip{w}\lambda\Zip{y}.x\ \Zip{\lambda\Zip{z}.z\Zip{w}}\ \Zip{y}$
\item $\lambda\Zip{w}\lambda\Zip{y}.w\ \Zip{\lambda\Zip{z}.x\ \Zip{z}\ \Zip{y}}$
\begin{footnotesize}
\item $\nd{\Zip{A\lolli B\lolli C}}{((A\lolli C)\lolli C)\lolli B\lolli C}$
\[ 
\infer[\lolli I]{\nd{\Zip{A\lolli B\lolli C}}{((A\lolli C)\lolli C)\lolli B\lolli C}}
{\infer[\lolli I]{\nd{\Zip{A \lolli B \lolli C},((A \lolli C)\lolli C)}{B \lolli C}}
{\infer[\lolli E]{\nd{\Zip{A \lolli B \lolli C},((A \lolli C)\lolli C),B}{C}}{{\infer[]{\nd{((A \lolli C) \lolli C)}{((A \lolli C) \lolli C)}}{Ax}}&
{\infer[\lolli I]{\nd{\Zip{A \lolli B \lolli C},B} {(A \lolli C)}}
{\infer[\lolli E]{\nd{\Zip{A \lolli B \lolli C},B,A}{C}}{
{\infer[\lolli E]{\nd{\Zip{A \lolli B \lolli C},A}{B \lolli C}}
{
{\infer[]{\nd{\Zip{A \lolli B \lolli C}}{\Zip{A \lolli B \lolli C}}}{Ax}}
&
{\infer[]{\nd{A}{A}}{Ax}}
}}
&
{\infer[]{\nd{B}{B}}{Ax}}
}}}}}}
\]
\[
\infer[\lolli I]{\nd{\Zip{x:((A \lolli C)\lolli C)\lolli B \lolli C}}{\lambda\Zip{w}.\lambda\Zip{y}.x\Zip{\lambda\Zip{z}.z\Zip{w}}.\Zip{y}}:A\lolli B \lolli C}
{\infer[\lolli I]{\nd{\Zip{x:((A \lolli C)\lolli C)\lolli B\lolli C},\Zip{w:A}}
{\lambda\Zip{y}.x.\Zip{\lambda\Zip{z}.z\Zip{w}}.\Zip{y}:B \lolli C}}
{\infer[\lolli E]{\nd{{\Zip{x:((A \lolli C)\lolli C)\lolli B\lolli C},\Zip{w:A},\Zip{y:B}}{{x\Zip{\lambda\Zip{z}.z\Zip{w}}.\Zip{y}}}:C}}{{s}&{g}}}
}
\]

\end{footnotesize}
\begin{footnotesize}
\item 
$\nd{\Zip{((A\lolli C)\lolli C)\lolli B\lolli C}}{A\lolli B\lolli C}$
\[
\infer[\lolli I]{\nd{\Zip{((A\lolli C)\lolli C)\lolli B\lolli C}}{A\lolli B\lolli C}}
{\infer[\lolli I]{\nd{\Zip{((A \lolli C)\lolli C)\lolli B \lolli C},A}{B \lolli C}}
{\infer[\lolli E]{\nd{\Zip{((A \lolli C)\lolli C)\lolli B \lolli C},A,B}{C}}{
{\infer[\lolli E]{\nd{\Zip{((A \lolli C)\lolli C)\lolli B \lolli C},A }{B \lolli C}}
{{\nd{\Zip{((A \lolli C)\lolli C)\lolli B \lolli C}}{\Zip{((A \lolli C)\lolli C)\lolli B \lolli C}}}&
{\infer[\lolli I]{\nd{A}{(A \lolli C)\lolli C}}
{\infer[\lolli E]{\nd{A,A \lolli C}{C}}
{{\infer[]{\nd{A \lolli C}{A\lolli C}}{Ax}}
&{\infer[]{\nd{A}{A}}{Ax}}
}
}}}}
&
{\infer[]{\nd{B}{B}}{Ax}}
}}}
\]
\end{footnotesize}
\end{enumerate}

Now consider the following translation from implication types/terms in
ordered grammar logic into linear implication types/terms.
The translation undoes the distinction between left and right implication:
both slashes are mapped to linear implication.
\[\W{types:}\qquad X^{*}=X,\quad (A/B)^{*} = (B\bs A)^{*} = B^{*}\lolli A^{*}\]
\[\W{terms:}\qquad
x^{*}=x,\quad (\llambda x.t)^{*}=(\rlambda x.t)^{*}=\lambda\Zip{x^{*}}.t^{*},\quad
(t\triangleleft u)^{*}=(u\triangleright t)^{*} = t^{*}\Zip{u^{*}}\]

One of the ILL judgements/terms above is the translation of a proof in ordered grammar logic \textbf{NL}
(no structural rules).
Give such a proof, using the rules as given in Exercises PH 5, in Blackboard. The other judgement/term does not
have a source with directional implications $\slash, \bs$ instead of $\lolli$.
Give the ILL proof for that term, showing explicitly where you have to
apply Exchange steps.

Hint: for the ILL type $A\lolli B\lolli C$, you will want to try its four ordered versions:
$A\bs(B\bs C)$, $A\bs(C/B)$, $(B\bs C)/A$, $(C/B)/A$. Similarly for the other types \ldots



\section{}

Let $\LK^-$ be the system consisting of the axioms and rules of $\LK$ minus Cut and the weakening and contraction rules. Give a formula $A$ such that the sequent $(\,\seq A)$ is provable in $\LK$, but not in $\LK^-$ and provide a  proof of the sequent in  $\LK$ and an argument showing its  nonprovability in $\LK^-$. 


\section{}


{\bf Proof of (II):}
The antecedent of the endsequent contains a implication, $A \imp B$, and the last inference is L$\en$ and thus looks as follows. 
\[
 \infer[L \en]{\Ga, C\en D \seq A \imp B,\De}{\Ga,C,D \seq A\imp B, \De} 
\]
By soundness, $I(\Ga, C\en D \seq A \imp B,\De)$ holds in $\cpc$. Therefore also 
$I(\Ga ,C \en D ,A \seq B, \De)$ holds in $\cpc$, as they follow from $I(\Ga, C\en D \seq A \imp B,\De)$. By completeness they are derivable in \Gth, say with derivation $\cald_C$. Then the following is a proof of $\Ga, C\en D \seq A \imp B,\De$ in which the last inference is an application of rule R$\imp$. 
\[
 \infer[R \imp]{\Ga, C\en D \seq A \imp B,\De}{\deduce[\cald_C]
 {\Ga, C\en D \seq A \imp B,\De}{} }
\]
 
{\bf Proof of (III):} 
The antecedent of the endsequent contains a implication, $C \imp D$, and the last inference is L$\en$ and thus looks as follows. 
\[
 \infer[L \en]
 {\Ga,C \imp D,A \en B \seq \De}
 {\Ga ,C \imp D, A,B \seq \De} 
\]
By soundness, $I(\Ga,C \imp D,A \en B \seq \De)$ holds in $\cpc$. Therefore also 
$I(\Ga , A \en B \seq C,\De)$ and $I(\Ga , A \en B,D \seq \De)$ holds in $\cpc$, as they follow from $I(\Ga,C \imp D,A \en B \seq \De)$. By completeness they are derivable in \Gth, say with derivations $\cald_C$ and $\cald_D$, respectively. Then the following is a proof of $\Ga,C \imp D,A \en B \seq \De$ in which the last inference is an application of rule L$\imp$. 
\[
 \infer[R \imp]{\Ga, C\en D \seq A \imp B,\De}{\deduce[\cald_C]
 {\Ga, A \en B \seq C,\De}
 {} & \deduce[\cald_d]
 {\Ga,A \en B ,D \seq \De}
 {} }
\]


\section{}

\[
\infer[R \en]{p \en (q\imp r) \seq p \en r}{
{\infer[L \en]{p \en (q \imp r)\seq p}
{\infer[]{p,(q \imp r)\seq p}{Ax}}
}&
{\infer[L \en]{p \en (q \imp r)\seq r}
{\infer[]{p,(q \imp r)\seq r}{{p \seq q,r}&
{\infer[]{p,r \seq r}{Ax}}
}}}}
\] 
The label S0 of  $p \seq q,r$ leaf is not an axiom. Let v be
the valuation that maps all atoms that occur in the antecedent of S0 to 1 and
all atoms that occur in the succedent of S0 to 0.
With induction to the depth of the sequent that for all sequents S1 along
the branch from that leaf to the root of the tree, v(I(S1)) = 0. Therefore
v(I(S)) = 0, and thus I(S) is not a tautology. 

\clearpage 

\end{document}
      
