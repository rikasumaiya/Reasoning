\documentclass[11pt]{article}

\usepackage[a4paper]{geometry}
\usepackage{amssymb} 
\usepackage{amsmath} 
\usepackage{euscript} 
\usepackage{latexsym}
\usepackage{mathrsfs}
\usepackage{stmaryrd}
\usepackage{graphicx}
\usepackage{dsfont}
\usepackage{proof}
\usepackage{color}
\usepackage{etex,xy}
\usepackage{version}
\xyoption{all} 

\includeversion{question}
\includeversion{solution}

\newcommand{\ipc}{{\sf IPC}}
\newcommand{\cpc}{{\sf CPC}}
\newcommand{\iqc}{{\sf IQC}}
\newcommand{\cqc}{{\sf CQC}}
\newcommand{\LJ}{{\sf LJ}}
\newcommand{\Gth}{{\sf G3}}
\newcommand{\GKp}{{\sf GKp}}
\newcommand{\LK}{{\sf LK}}
\newcommand{\K}{{\sf K}}
\newcommand{\KT}{{\sf KT}}
\newcommand{\Kf}{{\sf K4}}
\newcommand{\GL}{{\sf GL}}
\newcommand{\Sf}{{\sf S4}}

\newcommand{\T}{{T}}
\newcommand{\lang}{{\cal L}}
\newcommand{\lgc}{{\sf L}}

\newcommand{\langprop}{{\cal L}_{\rm prp}}
\newcommand{\langpred}{{\cal L}_{\rm prd}}
\newcommand{\formprop}{{\cal F}_{\rm prp}}
\newcommand{\formpred}{{\cal F}_{\rm prd}}

\newcommand{\af}{\vdash}
\newcommand{\waar}{\vDash}
\newcommand{\defn}{\equiv}
\newcommand{\slot}{\trs{$\Box$}}

\newcommand{\cala}{{\cal A}}
\newcommand{\calc}{{\cal C}}
\newcommand{\cald}{{\cal D}}
\newcommand{\mg}{{\cal G}}
\newcommand{\mh}{{\cal H}}
\newcommand{\cali}{{\cal I}}
\newcommand{\calp}{{\cal P}}
\newcommand{\calr}{{\cal R}}
\newcommand{\cals}{{\cal S}}
\newcommand{\calv}{{\cal V}}
\newcommand{\Ga}{\Gamma}
\newcommand{\The}{\Theta}  
\newcommand{\De}{\Delta} 
\newcommand{\Sig}{\Sigma}  
\newcommand{\ph}{\varphi}
\newcommand{\sig}{\sigma}

\newcommand{\imp}{\rightarrow} 
\newcommand{\Imp}{\, \Rightarrow \, }
\newcommand{\ifff}{\leftrightarrow} 
\newcommand{\Ifff}{\, \Leftrightarrow \, }
\newcommand{\Rgt}{\Rightarrow}
\newcommand{\Lft}{\Leftarrow}
\newcommand{\en}{\wedge} 
\newcommand{\of}{\vee} 
\newcommand{\E}{\exists}
\newcommand{\A}{\forall} 
\newcommand{\bof}{\bigvee}
\newcommand{\ben}{\bigwedge}
\newcommand{\bx}{\Box}
\newcommand{\dbx}{\boxdot}
\newcommand{\seq}{\Rightarrow}
\newcommand{\klg}{\leqslant}

\newcommand{\nd}[2]{#1 \vdash #2}
\newcommand{\sq}[2]{#1 \Rightarrow #2}
\newcommand{\Ra}{\rightarrow}
\newcommand{\Zip}[1]{\langle #1 \rangle}
\newcommand{\bang}{\mathop{!}}
\newcommand{\lolli}{\multimap}
\newcommand{\bs}{\backslash}
\newcommand{\with}{\mathbin{\&}}
\newcommand{\amp}{\mathbin{\&}}
\newcommand{\W}[1]{\textrm{#1}}
\newcommand{\fdia}{\diamondsuit}
\newcommand{\gbox}{\Box}
\newcommand\rlambda{\overset{{}_{\shortrightarrow}}{\lambda}}
\newcommand\llambda{\overset{{}_{\shortleftarrow}}{\lambda}}

\parindent0pt

\begin{document}
\setlength{\unitlength}{1cm}

\begin{center}
{\bf Methods in AI Research} \\
\vskip5pt 
{\bf Exercises II} \\
\vskip5pt 
\end{center}

\section{}
\begin{scriptsize}
$(\dag)\qquad$ 
\[
\infer[\lolli E]{\nd{\Zip{A},\Zip{B},\Zip{A\lolli (B\lolli C)}}{C}}
{\infer[\lolli I]{\nd{\Zip{A\lolli (B\lolli C)}}{(A\otimes B)\lolli C}}
{\infer[\otimes E]{\nd{\Zip{A \lolli (B \lolli C)},(A \otimes B)}{C}}{{\infer[]{\nd{(A \otimes B)}{(A \otimes B)}}{Ax}} & 
{\infer[\lolli E]{\nd{\Zip{A \lolli (B \lolli C)},{A},{B}}{C}}{
{\infer[\lolli E]{\nd{\Zip{A \lolli (B \lolli C)},A}{(B \lolli C)}}
{
{\infer[]{\nd{\Zip{A\lolli(B \lolli C)}}{\Zip{A \lolli (B \lolli C)}}}{Ax}}&{\nd{A}{A}}
}}
&{\infer[]{\nd{B}{B}}{Ax}}}}}} &
\infer[]{\nd{\Zip{A},\Zip{B}}{A\otimes B}}
{{\infer[]{\nd{\Zip{A}}{A}}{Ax}} &
{\infer[]{\nd{\Zip{B}}{B}}{Ax}}}}
\]

\end{scriptsize}

\begin{itemize}
  \item[-] the proof is not in normal form: perform the applicable
  proof reduction(s) to bring it in normal form;
  \begin{scriptsize}
\[
\infer[\lolli E]{\nd{\Zip{A},\Zip{B},\Zip{A \lolli (B \lolli C)}}{C}}
{{\infer[\lolli E]{\nd{\Zip{A \lolli (B \lolli C)},\Zip{A}}{(B \lolli C)}}{{\infer[]{\nd{\Zip{A\lolli(B \lolli C)}}{\Zip{A \lolli (B \lolli C)}}}{Ax}}&{\nd{\Zip{A}}{A}}}}&
{\infer[]{\nd{B}{B}}{Ax}}}
\]
\end{scriptsize}
  \item[-] compute the term for (\dag), and the term that results from term reduction(s).
\begin{center}
\begin{scriptsize}
\rotatebox{90}{
\infer[\lolli E]{\nd{\Zip{v:A},\Zip{w:B},\Zip{u:A\lolli (B\lolli C)}}
{\lambda\Zip{t}.case t of \Zip{v,w}\rightarrow u\Zip{t}\Zip{w}\Zip{\Zip{w:B}}:C}
}
{\infer[\lolli I]{\nd{u:\Zip{A\lolli (B\lolli C)}}{\lambda\Zip{t}.case t of \Zip{v,w}\rightarrow u\Zip{t}\Zip{w}:(A\otimes B)\lolli C}}
{\infer[\otimes E]{\nd{\Zip{u:A \lolli (B \lolli C)},(t:A \otimes B)}{case t of \Zip{v,w}\rightarrow u\Zip{t}\Zip{w}:C}}{{\infer[]{\nd{(A \otimes B)}{(A \otimes B)}}{Ax}} & 
{\infer[\lolli E]{\nd{\Zip{u:A \lolli (B \lolli C)},\Zip{v:A},\Zip{w:B}}{u\Zip{t}\Zip{w}:C}}{
{\infer[\lolli E]{\nd{\Zip{u:A \lolli (B \lolli C)},\Zip{v:A}}{u\Zip{t}:(B \lolli C)}}
{
{\infer[]{\nd{\Zip{u:A\lolli(B \lolli C)}}{\Zip{u:A \lolli (B \lolli C)}}}{Ax}}&{\nd{\Zip{v:A}}{v:A}
}}}
&{\infer[]{\nd{w:B}{w:B}}{Ax}}}}}} &
\infer[]{\nd{\Zip{v:A},\Zip{w:B}}{\Zip{v,w}:A\otimes B}}
{{\infer[]{\nd{\Zip{v:A}}{v:A}}{Ax}} &
{\infer[]{\nd{\Zip{w:B}}{w:B}}{Ax}}}}
}
\end{scriptsize}
\end{center}
\[
\infer[\lolli E]{\nd{\Zip{y:A},\Zip{z:B},\Zip{x:A \lolli (B \lolli C)}}{x\Zip{y}\Zip{z}: C}}
{{\infer[\lolli E]{\nd{x:\Zip{A \lolli (B \lolli C)},\Zip{y:A}}{x\Zip{y}:(B \lolli C)}}{{\infer[]{\nd{\Zip{x:A\lolli(B \lolli C)}}{x:\Zip{A \lolli (B \lolli C)}}}{Ax}}&{\nd{\Zip{y:A}}{y:A}}}}&
{\infer[]{\nd{z:B}{z:B}}{Ax}}}
\]
\end{itemize}
You may leave applications of Exchange implicit, i.e.~treat the antecedent as
a multiset of assumptions. Remember: the normal form is reached when \emph{no further
reductions} are applicable.


\section{}
Below you find two ILL judgements and two programs (terms). Which 
term goes with which judgement?

\begin{enumerate}
\item $\lambda\Zip{w}\lambda\Zip{y}.x\ \Zip{\lambda\Zip{z}.z\Zip{w}}\ \Zip{y}$
\item $\lambda\Zip{w}\lambda\Zip{y}.w\ \Zip{\lambda\Zip{z}.x\ \Zip{z}\ \Zip{y}}$
\begin{footnotesize}
\item $\nd{\Zip{A\lolli B\lolli C}}{((A\lolli C)\lolli C)\lolli B\lolli C}$
\begin{center}
\rotatebox{90}{
\begin{scriptsize}
\infer[\lolli I]{\nd{\Zip{x:A \lolli B \lolli C}}{\lambda\Zip{w}.\lambda\Zip{y}w\Zip{\lambda\Zip{z}.x.\Zip{z}\Zip{y}}:((A \lolli C)\lolli C)\lolli B \lolli C}}
{\infer[\lolli I]{{\nd{\Zip{w:((A\lolli C)\lolli C)\lolli B\lolli C},\Zip{x:A \lolli B \lolli C}}
{\lambda\Zip{y}w\Zip{\lambda\Zip{z}.x.\Zip{z}\Zip{y}}:B \lolli C}}}
{\infer[\lolli E]{\nd{\Zip{w:((A\lolli C)\lolli C)\lolli B\lolli C},\Zip{x:A \lolli B \lolli C},\Zip{y:B}}{w\Zip{\lambda\Zip{z}.x.\Zip{z}\Zip{y}}: C}}
{
{\nd{\Zip{w:((A\lolli C)\lolli C)\lolli B\lolli C}}{\Zip{w:((A\lolli C)\lolli C)\lolli B\lolli C}}}
&
{\infer[\lolli I]{\nd{\Zip{x:A \lolli B \lolli C},\Zip{y:B}}{\lambda\Zip{z}.x.\Zip{z}\Zip{y}:A \lolli C}}
{\infer[\lolli E]{\nd{\Zip{x:A \lolli B \lolli C},\Zip{z:A},\Zip{y:B}}{x\Zip{z}\Zip{y}:C}}
{
{\infer[\lolli E]{\nd{\Zip{x:A \lolli B \lolli C},\Zip{z:A}}{x\Zip{z}:B \lolli C}}
{
{\nd{\Zip{x:A \lolli B \lolli C}}{x:A \lolli B \lolli C}}
&{\nd{\Zip{z:A}}{z:A}}}
}
&{\nd{\Zip{y:B}}{y:B}}}
}
}
}
}}
\end{scriptsize}
}
\end{center}
This judgement matches with $\lambda\Zip{w}\lambda\Zip{y}.w\ \Zip{\lambda\Zip{z}.x\ \Zip{z}\ \Zip{y}}$ term.
\end{footnotesize}
\item
$\nd{\Zip{((A\lolli C)\lolli C)\lolli B\lolli C}}{A\lolli B\lolli C}$
\begin{center}
\rotatebox{90}{
\begin{scriptsize} 
\infer[\lolli I]{\nd{\Zip{x:((A\lolli C)\lolli C)\lolli B \lolli C}} \lambda\Zip{w}.\lambda\Zip{y}.x\Zip{\lambda\Zip{z}.z\Zip{w}}.\Zip{y}:A\lolli B \lolli C}
{\infer[\lolli I]{\nd{\Zip{x:((A\lolli C)\lolli C)\lolli B \lolli C},\Zip{w:A}}{\lambda\Zip{y}.x\Zip{\lambda\Zip{z}.z\Zip{w}}.\Zip{y}:B \lolli C}}
{\infer[\lolli E]{\nd{\Zip{x:((A\lolli C)\lolli C)\lolli B \lolli C},\Zip{w:A},\Zip{y:B}}{x\Zip{\lambda\Zip{z}.z\Zip{w}}.\Zip{y}: C}}
{
{\infer[\lolli E]{\nd{\Zip{x:((A\lolli C)\lolli C)\lolli B \lolli C},\Zip{w:A}}{x\Zip{\lambda\Zip{z}.z\Zip{w}}:B \lolli C}}
{{\nd{\Zip{x:((A\lolli C)\lolli C)\lolli B \lolli C}}{x:((A\lolli C)\lolli C)\lolli B \lolli C}}&
{\infer[\lolli I]{\nd{\Zip{w:A}}{\lambda\Zip{z}.z\Zip{w}:((A \lolli C)\lolli C)}}
{\infer[\lolli E]{\nd{\Zip{z:A\lolli C},\Zip{w:A}}{z\Zip{w}:C}}
{{\nd{\Zip{z:A \lolli C}}{z:A \lolli C}}&{\nd{\Zip{w:A}}{w:A}}}
}}}}&{\nd{\Zip{y:B}}{y:B}}}
}}
\end{scriptsize}
}
\end{center}
This judgement matches with $\lambda\Zip{w}\lambda\Zip{y}.x\ \Zip{\lambda\Zip{z}.z\Zip{w}}\ \Zip{y}$ term.
\end{enumerate}

\section{}

\[
\infer[R \of]{A,A \imp B \seq A \of B}
{\infer[L \imp]{A,A \imp B \seq A,B}{
{\infer[RW]{A,A \seq A,B}{\infer[LW]{A,A \seq A}{\infer[]{A \seq A}{Ax}}}}
&
{\infer[RW]{A,B \seq A,B}
{\infer[LW]{A,B \seq B}{\infer[]{B \seq B}{Ax}}}
}
}}
\]

For the above proof without the weakening rule we cannot reach axioms in LK system.Therefore in the $LK^-$ this sequent is non provable.

\section{}


{\bf Proof of (II):}
The antecedent of the endsequent contains a implication, $A \imp B$, and the last inference is L$\en$ and thus looks as follows. 
\[
 \infer[L \en]{\Ga, C\en D \seq A \imp B,\De}{\Ga,C,D \seq A\imp B, \De} 
\]
By soundness, $I(\Ga, C\en D \seq A \imp B,\De)$ holds in $\cpc$. Therefore also 
$I(\Ga ,C \en D ,A \seq B, \De)$ holds in $\cpc$, as they follow from $I(\Ga, C\en D \seq A \imp B,\De)$. By completeness they are derivable in \Gth, say with derivation $\cald_C$. Then the following is a proof of $\Ga, C\en D \seq A \imp B,\De$ in which the last inference is an application of rule R$\imp$. 
\[
 \infer[R \imp]{\Ga, C\en D \seq A \imp B,\De}{\deduce[\cald_C]
 {\Ga, C\en D \seq A \imp B,\De}{} }
\]
 
{\bf Proof of (III):} 
The antecedent of the endsequent contains a implication, $C \imp D$, and the last inference is L$\en$ and thus looks as follows. 
\[
 \infer[L \en]
 {\Ga,C \imp D,A \en B \seq \De}
 {\Ga ,C \imp D, A,B \seq \De} 
\]
By soundness, $I(\Ga,C \imp D,A \en B \seq \De)$ holds in $\cpc$. Therefore also 
$I(\Ga , A \en B \seq C,\De)$ and $I(\Ga , A \en B,D \seq \De)$ holds in $\cpc$, as they follow from $I(\Ga,C \imp D,A \en B \seq \De)$. By completeness they are derivable in \Gth, say with derivations $\cald_C$ and $\cald_D$, respectively. Then the following is a proof of $\Ga,C \imp D,A \en B \seq \De$ in which the last inference is an application of rule L$\imp$. 
\[
 \infer[R \imp]{\Ga, C\en D \seq A \imp B,\De}{\deduce[\cald_C]
 {\Ga, A \en B \seq C,\De}
 {} & \deduce[\cald_d]
 {\Ga,A \en B ,D \seq \De}
 {} }
\]


\section{}

\[
\infer[R \en]{p \en (q\imp r) \seq p \en r}{
{\infer[L \en]{p \en (q \imp r)\seq p}
{\infer[]{p,(q \imp r)\seq p}{Ax}}
}&
{\infer[L \en]{p \en (q \imp r)\seq r}
{\infer[]{p,(q \imp r)\seq r}{{p \seq q,r}&
{\infer[]{p,r \seq r}{Ax}}
}}}}
\] 
The label S0 of  $p \seq q,r$ leaf is not an axiom. Let v be
the valuation that maps all atoms that occur in the antecedent of S0 to 1 and
all atoms that occur in the succedent of S0 to 0.
With induction to the depth of the sequent that for all sequents S1 along
the branch from that leaf to the root of the tree, v(I(S1)) = 0. Therefore
v(I(S)) = 0, and thus I(S) is not a tautology. 

\clearpage 

\end{document}
      
